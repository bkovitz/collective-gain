\documentclass{article}
\begin{document}

% ABSTRACT

Selective pressure can favor a gene for altruistic behavior even in a
community of takers who give nothing back, even when the behavior results in
the giver having no offspring, even when there is no mechanism to detect or
punish takers who give nothing back, and even when givers must compete with
their own beneficiaries for the resources needed for giving. Three conditions
suffice to produce this selective pressure: (1) the gene encodes only a
probability of engaging in the giving behavior; (2) benefits from the behavior
decrease with distance from the giver; and (3) organisms can produce more than
one offspring. A population under these conditions tends to contain a stable
range of variation in alleles that code for different probabilities of the
giving behavior, in some cases producing castes centered at different
probabilities. In effect, alleles for varying degrees of giving produce
partial clones who are partially sterile in proportion to their giving. Such
alleles represent a range of variation between the sterile castes of social
insects, or somatic cells in multicellular organisms (100\% clones, 100\%
givers, 100\% sterile), and organisms that try to maximize inclusive fitness
entirely through their own offspring. The selective pressure favoring partial
givers provides a genetic explanation for some forms of group selection, and
predicts as-yet-unexplored correlations between communal giving and sterility.

\end{document}
